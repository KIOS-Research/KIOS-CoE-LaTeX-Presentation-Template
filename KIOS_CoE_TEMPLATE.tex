\documentclass[hyperref={pdfpagelabels=false},aspectratio=169]{beamer}
%\documentclass[hyperref={pdfpagelabels=false},aspectratio=43]{beamer} 
\let\Tiny=\tiny
\usepackage[T1]{fontenc} 
\usepackage[utf8]{inputenc}


\usetheme{kios}

\definecolor{red}{rgb}{0.8235,0.1294,0.2039}
\definecolor{kiosorange}{rgb}{0.9686,0.5765,0.1137}
\definecolor{kiosgray}{rgb}{0.5020,0.5098,0.5216}
\definecolor{kiosyellow}{rgb}{0.9725,0.9216,0.12552}
\setbeamercolor{part name}{fg=kiosorange}
\setbeamercolor{part title}{fg=red}
\setbeamercolor{button}{use=local structure,bg=kiosorange,fg=red}
\setbeamercolor{button border}{use=button,fg=button.bg}
\usepackage{algorithm,algpseudocode}


\setbeamertemplate{navigation symbols}{} % Remove control butttons from the bottom


\usepackage{fancybox}

\usepackage{tabto} % for the \tab command
\usepackage[utf8]{inputenc}
\usepackage[english]{babel}
\usepackage{amsmath}
\usepackage{amsfonts}
\usepackage{amssymb}
\usepackage{graphicx}
\usepackage{xspace}

\usepackage{listings} % PACKAGE FOR CODE INPUT IN DOCUMENT

\usepackage{expl3}
\ExplSyntaxOn
\int_zero_new:N \g__prg_map_int 
\ExplSyntaxOff

\usepackage{tikz}
\usetikzlibrary{tikzmark,decorations.pathreplacing,calligraphy,decorations.markings,arrows.meta,shapes.arrows}

\newcommand{\latex}{\LaTeX\xspace}
\newcommand{\red}[1]{\textcolor{red}{#1}}

\setbeamertemplate{bibliography item}[text]


% ADD THE FOLLOWING COUPLE LINES INTO YOUR PREAMBLE
\let\OLDthebibliography\thebibliography
\renewcommand\thebibliography[1]{
	\OLDthebibliography{#1}
	\setlength{\parskip}{-5pt}

}
\usepackage{ragged2e}
\usepackage{etoolbox}

\apptocmd{\frame}{}{\justifying}{} % Allow optional arguments after frame.


\bibliographystyle{plain}



\setbeamercovered{transparent}


\begin{document}


\title[] {Title of Presentation}

\author[]{\underline{First Author}\inst{1} \and 
		Second Author\inst{2} \and 
		Third Author\inst{1} \and  
		Fourth Author\inst{1}}

\institute[]{\inst{1} KIOS Research and Innovation Center of Excellence and the Department of Electrical and Computer Engineering, University of Cyprus, Cyprus \and \inst{2} Department of Electrical and Computer Engineering, Imperial College, London}%

\subtitle{Conference Title}

\date{Date of presentation}


\begin{frame}[fragile,plain]
	\maketitle
\end{frame}


\begin{frame}
	Part I: Title of Part 1
	
	\tableofcontents[part=1,hideallsubsections]
	Part II: Title of Part 2
	\tableofcontents[part=2,hideallsubsections]
\end{frame}


%\setbeamertemplate{section in toc}[square]
%\begin{frame}{Outline}
%	 \tableofcontents[hideallsubsections]
%\end{frame}

\part{Title of Part 1}
\frame{\partpage}
\section{Introduction}
\begin{frame}
	\frametitle{There Is No Largest Prime Number} 
	\framesubtitle{The proof uses \textit{reductio ad absurdum}.} 
	\begin{theorem}
		There is no largest prime number.  \end{theorem} 
	\begin{enumerate} 
		\item<1-| alert@1> Suppose $p$ were the largest prime number. 
		\item<2-> Let $q$ be the product of the first $p$ numbers. 
		\item<3-> Then $q+1$ is not divisible by any of them. 
		\item<1-> But $q + 1$ is \color{KIOSred}greater\color{KIOSgray}\ than $1$, thus divisible by some prime
		number not in the first $p$ numbers.
		
	\end{enumerate}
	
	\begin{itemize}
		\item one
		\item two 
		\begin{itemize}
			\item three
			\item four
		\end{itemize}
	\end{itemize} 
\end{frame}

\part{Title of Part 2}
\frame{\partpage}
\section{Problem Formulation}
\begin{frame} {Block Examples}
	\begin{alertblock}{Alert Block}
		\begin{itemize}
			\item Item 1
			\item Item 2
			\item \alert{Alerted text}
	\end{itemize}	
\end{alertblock}
	\begin{example}{Example Block}
	\begin{itemize}
		\item Item 1
		\item Item 2
		\item \alert{Alerted text}
	\end{itemize}	
\end{example}

\end{frame}





\end{document}